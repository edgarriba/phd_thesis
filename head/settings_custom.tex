%%%%%%%%%%%%%%%%%%%%%%%%%%%%%%%%%%%%%%%%%%%%%%
%
%		Thesis Settings
%		Custom settings
%
%		2011
%
%%%%%%%%%%%%%%%%%%%%%%%%%%%%%%%%%%%%%%%%%%%%%%

%
%   Use this file for your own custom packages, command-definitions, etc...
%
\usepackage{enumitem}		% enumerate with letters
\usepackage{colortbl}
\usepackage{nccmath}
\usepackage{bibentry}
\usepackage{filecontents}
\usepackage{epstopdf}
\usepackage{ragged2e}
\usepackage[utf8]{inputenc}
\usepackage{textcomp}
\usepackage[english]{babel}
\usepackage{bm}  
\usepackage{amsfonts}   
\usepackage{verbatim}
\usepackage{amssymb}
\usepackage{graphics}
\usepackage{algorithm}
\usepackage{algorithmic}
\usepackage{cite}
%\usepackage{fix2col}
\usepackage{color}
\usepackage{geometry}
\usepackage{longtable}
\usepackage{amsmath}
\usepackage{pifont}
\usepackage{lscape}
\usepackage{xspace}
\usepackage{amsopn}
\usepackage{mathtools}
\usepackage{etextools}

\usepackage{float}
%%\usepackage{mathptmx}        % selects Times Roman as basic font
%\usepackage{helvet}          % selects Helvetica as sans-serif font
%\usepackage{courier}         % selects Courier as typewriter font
\usepackage{type1cm}         % activate if the above 3 fonts are not available on your system
\usepackage{makeidx}         % allows index generation
%\usepackage{graphicx}        % standard LaTeX graphics tool when including figure files
\usepackage{multicol}        % used for the two-column index
\usepackage[bottom]{footmisc}% places footnotes at page bottom
%\usepackage[hidelinks, breaklinks]{hyperref}
%\usepackage{epsfig}
%\usepackage{graphicx}
\usepackage{amsmath}
\usepackage{amssymb}
\usepackage{bm}
\usepackage{mathtools}
\usepackage{multirow}
\usepackage{array}
\usepackage{url}
\usepackage{lipsum}
\usepackage{dblfloatfix}
\usepackage{setspace}
\usepackage{tabulary}
\usepackage{natbib}
\usepackage{xstring}

% for kornia paper
\usepackage{minted}  % for kornia snipets
\usepackage{blindtext}
\usepackage{geometry}
\usepackage{float}
% \usepackage[italian]{babel}
\usepackage[most]{tcolorbox}
\usepackage{tikz}
\usepackage{varwidth}
\usepackage{capt-of}
\definecolor{bg}{rgb}{0.95,0.95,0.95}

% kornia dda paper
\usepackage{makecell}
\captionsetup[table]{skip=2pt} % adjust caption spacing
\usepackage{soul} % crossout text
\usepackage[export]{adjustbox} % adjust includegraphics
\usepackage[]{authblk}

% keynet paper
\usepackage{tabularx, booktabs}
\usepackage{subcaption}

% tfeat paper
\usepackage{adjustbox}
\usepackage{hyperref}
\usepackage{tikz}
\usepackage{bm}
\usetikzlibrary{shapes,arrows}
\usetikzlibrary{positioning}
\usetikzlibrary{decorations.pathreplacing}

 
\hyphenation{lo-ca-li-za-ti-on Lo-ca-li-za-ti-on}

% the following lines are for creating a simplified TO-DO box. However since boites is not per default installed with all latex-distributions, we have removed this example again
% if you want to use it and do not have "boites" installed, you can get it from here: http://www.ctan.org/tex-archive/macros/latex/contrib/boites
%
%\usepackage{boites,boites_exemples}
%\newcommand{\todolist}[1]{\begin{boiteepaisseavecuntitre}{TO DO in this chapter} #1 \end{boiteepaisseavecuntitre}}  % creates a little box
% %\newcommand{\todolist}[1]{}  % to be used when to do is not to be printed
\newcommand{\UAB}{Universitat Autònoma de Barcelona}
\newcommand{\CVC}{Centre de Visió per Computador}
\newcommand{\DCC}{Dept. Ciències de la computació}
\newcommand{\DCCCVC}{\DCC{} \& \CVC{}}
\newcommand{\printer}{Ediciones Gráficas Rey, S.L.}
\newcommand{\dedication}{A ellos}
\newcommand{\sentence}{\textit{The difficulty lies, not in the new ideas,}\\ \textit{but in escaping from the old ones}\\ John Maynard Keynes (1883 - 1946)}
\newcommand{\ISBN}{978-84-945373-1-8}

\newtheorem{thm}{Theorem}%[section]
\newtheorem{cor}[thm]{Corollary}
\newtheorem{lem}[thm]{Lemma}
\newtheorem{prop}[thm]{Proposition}
%\theoremstyle{definition}
\newtheorem{defn}[thm]{Definition}
%\theoremstyle{remark}
\newtheorem{rem}[thm]{Remark}
\newcommand{\tx}[1]{#1}
\newcommand{\ty}[1]{\tiny\bf #1}

\newcommand{\colR}[1]{\textcolor{red}{\bf #1}}
\newcommand{\colB}[1]{\textcolor{blue}{\bf #1}}

\newcommand{\minisection}[1]{\vspace{0.04in} \noindent {\bf #1}\ \ }

\newlength{\reducedwidth}
\setlength{\reducedwidth}{\textwidth}
\addtolength{\reducedwidth}{-\parindent}
\addtolength{\reducedwidth}{-\parindent}
\newenvironment{abstract}{\begin{center}\begin{minipage}{\reducedwidth}
\hrulefill\vspace{3pt}\bf\\\small}{\par\hrulefill\\\end{minipage}\end{center}}


\DeclareMathOperator*{\argmin}{\arg\!\min}
\DeclareMathOperator*{\argmax}{\arg\!\max}
\DeclarePairedDelimiter{\norm}{\lVert}{\rVert}

\newcommand\scalemath[2]{\scalebox{#1}{\mbox{\ensuremath{\displaystyle #2}}}}

\definecolor{light-gray}{gray}{0.4}

\def\ontop#1#2{\setbox0\hbox{#2}\copy0\llap{\raise\ht0\hbox{#1}}}

\providecommand{\e}[2]{\ensuremath{{#1}\text{e}{#2}}}

\newcommand\degrees[1]{#1\ensuremath{^\circ}}
\graphicspath{{figures/}}
\renewcommand{\baselinestretch}{1.0}

\newtheorem{theorem}{Theorem}[section]
\newtheorem{lemma}[theorem]{Lemma}
\newtheorem{proposition}[theorem]{Proposition}
\newtheorem{corollary}[theorem]{Corollary}

\newcommand{\mb}[1]{\mbox{$\underline{#1}$}}

\newenvironment{proof}[1][Proof]{\begin{trivlist}
\item[\hskip \labelsep {\bfseries #1}]}{\end{trivlist}}
\newenvironment{example}[1][Example]{\begin{trivlist}
\item[\hskip \labelsep {\bfseries #1}]}{\end{trivlist}}
\newenvironment{remark}[1][Remark]{\begin{trivlist}
\item[\hskip \labelsep {\bfseries #1}]}{\end{trivlist}}

\newcommand{\qed}{\nobreak \ifvmode \relax \else
      \ifdim\lastskip<1.5em \hskip-\lastskip
      \hskip1.5em plus0em minus0.5em \fi \nobreak
      \vrule height0.75em width0.5em depth0.25em \fi}


\newcommand{\Tr}[0]{$\mathcal{T}_r$ }
\newcommand{\mexp}[1]{$\textit{\bf exp}(#1)$ }
\newcommand{\bt}[1]{\tilde{\bar{#1}}}

\DeclareTextFontCommand{\emph}{\em}

\def\eg{\emph{e.g.~}} \def\Eg{\emph{E.g.~}}
\def\ie{\emph{i.e.}} \def\Ie{\emph{I.e.}}
\def\cf{\emph{c.f.}} \def\Cf{\emph{C.f.}}
\def\etc{\emph{etc.}} \def\vs{\emph{vs.}}
\def\wrt{w.r.t.} \def\dof{d.o.f.}
\def\etal{\emph{et al.}}

\newcommand{\fig}[1]{Fig.~\ref{#1}}
\newcommand{\Fig}[1]{Figure \ref{#1}}

\newcommand{\sect}[1]{Sect. \ref{#1}}
\newcommand{\Sect}[1]{Section \ref{#1}}
\newcommand{\ch}[1]{Chapt. \ref{#1}}
\newcommand{\Ch}[1]{Chapter \ref{#1}}
\newcommand{\app}[1]{App. \ref{#1}}
\newcommand{\tab}[1]{Table \ref{#1}}
\newcommand{\alg}[1]{Alg. \ref{#1}}

\newcommand{\subfig}[2]{Fig.~\ref{#1}#2}
\newcommand{\subtab}[2]{Table~\ref{#1}#2}
\newcommand{\points}{...,}
\newcommand{\figs}[2]{Fig. \ref{#1} and \ref{#2}}
%\newcommand{\eq}[1]{(\ref{#1})} %exigencias de ieee
\newcommand{\iii}{{\cal I}}
\newcommand{\bfp}{{\bf p}}
\newcommand{\Algorithm}[1]{Algorithm~\ref{#1}}
\newcommand{\mys}[1]{{\footnotesize{#1}}}
\newcommand{\degree}{\ensuremath{^\circ}}
\renewcommand{\labelenumii}{\theenumii}



\definecolor{light-gray}{gray}{0.4}

\def\ontop#1#2{\setbox0\hbox{#2}\copy0\llap{\raise\ht0\hbox{#1}}}

\providecommand{\e}[2]{\ensuremath{{#1}\text{e}{#2}}}




\def\mA{\mathcal{A}}
\def\mB{\mathcal{B}}
\def\mC{\mathcal{C}}
\def\mG{\mathcal{G}}
\def\mV{\mathcal{V}}
\def\mE{\mathcal{E}}
\def\mF{\mathcal{F}}
\def\mH{\mathcal{H}}
\def\mL{\mathcal{L}}
\def\mM{\mathcal{M}}
\def\mN{\mathcal{N}}
\def\mP{\mathcal{P}}
\def\mS{\mathcal{S}}
\def\mT{\mathcal{T}}
\def\mU{\mathcal{U}}
\def\mW{\mathcal{W}}
\def\mX{\mathcal{X}}
\def\mY{\mathcal{Y}}
\def\1n{\mathbf{1}_n}
\def\0{\mathbf{0}}
\def\1{\mathbf{1}}
\def\I{{\bf I}}

\newcommand{\UP}[1]{(\IfSubStr{#1}{-}{ \textcolor{red}{\bf #1} }{ \textcolor{blue}{\bf #1}})}
\newcommand{\BL}[1]{\textcolor{blue}{\bf #1}}
\newcommand{\OurNetwork}{CDNet}
\newcommand{\SEthree}{\ensuremath{\mathrm{SE}(3)}\xspace}

\makeatletter
\newsavebox\saved@arstrutbox
\newcommand*{\setarstrut}[1]{%
  \noalign{%
    \begingroup
      \global\setbox\saved@arstrutbox\copy\@arstrutbox
      #1%
      \global\setbox\@arstrutbox\hbox{%
        \vrule \@height\arraystretch\ht\strutbox
               \@depth\arraystretch \dp\strutbox
               \@width\z@
      }%
    \endgroup
  }%
}
\newcommand*{\restorearstrut}{%
  \noalign{%
    \global\setbox\@arstrutbox\copy\saved@arstrutbox
  }%
}
\makeatother

\newcommand{\GR}[1]{\textcolor{green}{\bf #1}}