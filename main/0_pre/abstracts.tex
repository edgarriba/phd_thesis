%\begingroup
%\let\cleardoublepage\clearpage



% English abstract
%\cleardoublepage
\chapter*{Abstract}

%\markboth{Abstract}{Abstract}
\addcontentsline{toc}{chapter}{Abstract (English/Spanish/Catalan)} % adds an entry to the table of contents
% put your text here
\vspace{-24mm}

Making Ground Autonomous Vehicles (GAVs) a reality as a service for the society is one of the major scientific and technological challenges of this century. The potential benefits of autonomous vehicles include reducing accidents, improving traffic congestion and better usage of road infrastructures, among others. These vehicles must operate in our cities, towns and highways, dealing with many different types of situations while respecting traffic rules and protecting human lives. GAVs are expected to deal with all types of scenarios and situations, coping with an uncertain and chaotic world. Therefore, in order to fulfil these demanding requirements GAVs need to be endowed with the capability of understanding their surrounding at many different levels, by means of affordable sensors and artificial intelligence. This capacity to understand the surroundings and the current situation that the vehicle is involved in, is called scene understanding. In this work we investigate novel techniques to bring scene understanding to autonomous vehicles by combining the use of cameras as the main source of information---due to their versatility and affordability---and algorithms based on computer vision and machine learning. We investigate different degrees of understanding of the scene, starting from basic geometric knowledge about \textit{where} is the vehicle within the scene. A robust and efficient estimation of the vehicle location and pose with respect to a map is one of the most fundamental steps towards autonomous driving. We study this problem from the point of view of robustness and computational efficiency, proposing key insights to improve current solutions. Then we advance to higher levels of abstraction to discover \textit{what} is in the scene, by recognizing and parsing all the elements present on a driving scene, such as roads, sidewalks, pedestrians, etc. We investigate this problem known as semantic segmentation, proposing new approaches to improve recognition accuracy and computational efficiency. We cover these points by focusing on key aspects such as: (i) how to leverage computation moving semantics to an offline process, (ii) how to train compact architectures based on deconvolutional networks to achieve their maximum potential, (iii) how to use virtual worlds in combination with domain adaptation to produce accurate models in a cost-effective fashion, and (iv) how to use transfer learning techniques to prepare models to new situations. We finally extend the previous level of knowledge enabling systems to reasoning about \textit{what has change} in a scene with respect to a previous visit, which in return allows for
efficient and cost-effective map updating.

\vspace{1mm}
\textbf{Key words:} \textit{autonomous driving, computer vision, machine learning, applied mathematics}

% German abstract

\begin{otherlanguage}{spanish}
%\cleardoublepage
\chapter*{Resumen}

\vspace{-24mm}

Hacer de los Vehículos Autónomos Terrestres una realidad al servicio de la sociedad supone uno de los mayores retos científicos de este siglo. Los beneficios potenciales de estos vehículos incluyen reducir accidentes, mejorar el tráfico y un mejor aprovechamiento de las infraestructuras. Los vehículos autónomos deben ser capaces de moverse en nuestras ciudades y autopistas, haciendo frente a cualquier tipo de situación mientras respetan las reglas de tráfico y protegen vidas humanas. Se requiere que estos vehículos se desenvuelvan en cualquier escenario, lidiando con información incierta y un mundo caótico. Para cumplir con esto,W los vehículos autónomos deben estar dotados de la capacidad para entender el entorno a diferentes niveles de complejidad, mediante el uso de sensores asequibles y de la inteligencia artificial. La capacidad de entender el medio en el que operan estos vehículos se conoce como entendimiento de la escena. En este trabajo se investigan nuevos técnicas para dotar a los vehículos de la capacidad para entender la escena, mediante la combinación de cámaras (debido a su versatilidad y bajo coste) y algoritmos de visión artificial y aprendizaje automático. Investigamos diferentes grados de entendimiento de la escena, comenzando por el conocimiento geométrico sobre \textit{dónde} está el vehículo con respecto a la escena. La estimación robusta y eficiente de la posición y pose del vehículo con respecto a un mapa es uno de los puntos fundamentales para la conducción autónoma. Por ello estudiamos el problema desde el punto de vista de la robustez y la eficiencia computacional, proponiendo ideas clave para mejorar las soluciones actuales. Tras ello, avanzamos hacia niveles de abstracción más altos, para descubrir \textit{qué} hay en la escena, reconociendo y segmentando todos los elementos de la misma, tales como: carreteras, aceras, peatones, etc. Investigamos este problema, conocido como segmentación semántica, proponiendo nuevas soluciones para mejorar el grado de acierto y la eficiencia del sistema. Para conseguir estos puntos nos centramos en aspectos clave, tal y como: (i) reducir el coste computacional estimando la semántica en un proceso \textit{offline}, (ii) cómo entrenar modelos neuronales compactos para extraer su máximo potencial, (iii) cómo usar mundos virtuales junto con técnicas de adaptación de dominio para producir modelos precisos y de forma más asequible, y (iv) cómo usar técnicas de transferencia de conocimiento para que los modelos puedan trabajar en nuevos entornos. Finalmente, extendemos las capacidades del sistema, capacitándolo para razonar sobre \textit{qué cosas han cambiado} en la escena con respecto a un tiempo anterior, lo que a cambio posibilita la actualización eficiente y barata de los mapas.

\vspace{1mm}
\textbf{Palabras clave:} \textit{vehículos autónomos, visión artificial, aprendizaje automático}



%put your text here
\end{otherlanguage}



% French abstract
\begin{otherlanguage}{catalan}
\cleardoublepage
\chapter*{Resum}
\vspace{-24mm}

Fer dels Vehicles Autònoms Terrestres una realitat al servei de la societat suposa un dels majors reptes científics i tecnològics d'aquest segle. Els beneficis potencials d'aquests vehicles inclouen reduir accidents, millorar el trànsit i un millor aprofitament de les infraestructures, entre molts d’altres. Els vehicles autònoms s’han de poder moure a les nostres ciutats, pobles i autopistes, enfrontant-se a qualsevol tipus de situació mentre respecten la normativa de trànsit i protegeixen vides humanes. Es requereix que aquests vehicles es desenvolupin en qualsevol escenari, bregant amb informació incerta i un món caòtic. Per complir aquests objectius els vehicles autònoms han d'estar dotats de la capacitat d’entendre l'entorn a diferents nivells de complexitat, mitjançant l'ús de sensors assequibles i de la intel·ligència artificial. La capacitat d'entendre el medi en el qual operen aquests vehicles es coneix com a enteniment de l'escena. En aquest treball s'investiguen noves tècniques per dotar els vehicles de la capacitat per entendre l'escena, mitjançant la combinació de càmeres (gràcies a la seva versatilitat i el seu baix cost) i algoritmes de visió artificial i d’aprenentatge automàtic. Vam investigar diferents graus d'enteniment de l'escena, començant pel coneixement geomètric d’on és el vehicle dins aquesta escena. L'estimació robusta i eficient de la posició i posi del vehicle respecte d’un mapa és un dels punts fonamentals per a la conducció autònoma. Per això estudiem el problema des dels punts de vista de la robustesa i de l'eficiència computacional, proposant idees clau per a millorar les solucions actuals. Després d'això, avancem cap a nivells d'abstracció més alts, per descobrir què hi ha a l'escena, reconeixent i segmentant tots els elements de la mateixa, com ara: carreteres, voreres, vianants, etc. Vam investigar aquest problema, conegut com a segmentació semàntica, proposant noves solucions per a millorar el grau d'encert i l'eficiència del sistema. Per aconseguir aquests punts ens centrem en aspectes clau, tals com: (i) reduir el cost computacional calculant la semàntica en un procés offline, (ii) entrenar models neuronals compactes per extreure el seu màxim potencial, (iii) fer servir móns virtuals juntament amb tècniques d'adaptació de domini per produir models precisos i de forma més assequible, i (iv) utilitzar tècniques de transferència i de coneixement perquè els models puguin treballar en nous entorns. Finalment, estenem les capacitats del sistema, instruint-lo per raonar sobre quines coses han canviat a l'escena respecte d’un temps anterior, fet que permet una actualització eficient i de baix cost dels mapes.

\vspace{1mm}
\textbf{Paraules clau:} \textit{vehicles autònoms, visió artificial, aprenentatge automàtic}


\end{otherlanguage}

%\endgroup			
%\vfill
