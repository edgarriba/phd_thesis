\setlength{\parindent}{15pt}
\setlength{\parskip}{0em}

\chapter*{Acknowledgements}
\vspace{-15mm}
This stage as a PhD student has been one of the most rewarding, complete and hard of my life. It has been a stage of challenges, fun and discovery. I am fully aware of the evolution that this stage has produced in me, and I am no less aware of the influence that many people have had on me. To get where I am and to perceive the world as I perceive it today. Shall this serve as a humble recognition of all those who contributed to this work. I hope not to forget anyone.

Let me start with those academically close, \ie, my supervisors, without their support this thesis would have never seen the light. Angel, thank you for your unconditional support, discussions and  the freedom you gave me. Antonio, thanks for letting me be part of something bigger, for considering me for the toughest projects and for your help in the worst moments. Julio, I would need too many pages to express you my gratitude. Your involvement and support in this thesis have been enormous. Thank you for having walked this path by my side and for teaching me so much; some of the best moments have been in front of your digital whiteboard. I would also like to thank Daniel Ponsa, for his opinions and ideas, which have helped to improve this project.

I also thank those who welcomed me into their research labs. Prof. Christoph Stiller at MRT, KIT. Jose Manuel Alvarez and Richard Hartley for the opportunity they gave me at NICTA. Pablo F. Alcantarilla and Bjorn Stenger for sharing their time with me at Toshiba Research Europe. Watanabe-san and Okada-san for giving me the opportunity to participate in the Visconti project. I also must thank, Andreas Geiger, Henning Lategahn, Simon Stent and Riccardo Gherardi, who during my stays abroad opened my eyes to new ideas and shared their time with me.

My deepest gratitude to my colleagues at the Computer Vision Center with whom I shared many moments. Allow me to begin with those who I had the pleasure to closely work with, such as David Vazquez, whose tenacity has inspired us all; Biel, Jordi, Sergi, Laura and Joanna, who allowed me to teach a little of what I know in exchange of how much they taught me. And of course, thanks to all the SYNTHIA team, especially to Laura, Fran, Elias and Marc, for their efforts and for helping me during this journey.

I must also thank administration staff of CVC for all the help and sympathy provided over the years. To Mari Carmen, Ana, Claire, Montse, Eva, Silvia, Mireia, Alexandra and Meritxell, thank you very much. I thank those who made my way through the CVC memorable. Thank you for providing hospitality and humanity: Marco Pedersoli, Pep Gonfaus, Javier Marin, Jordi Gonzalez, Xu Hu, Adela Barbulescu, Camp Davesa and Ivet Rafegas. Without Felipe, Hana, Yaxing, Dena, Victor and Bojana lunch breaks would not be the same. And of course I especially thank to my closest circle, Onur, Gemma, Ariel, Francesco, and Arash, who have been there with me, sharing good and bad times. You know well that I would never be able to pay back all what they have done for me. Your friendship has been the best of this stage.

I leave my most inner circle to the end. Thanks to all my family, my parents, Jose Angel and Maria Dolores, my sister Marina and my grandparents, Lola, Carmen, Manuel and Pedro. Thank you for all the strength and courage you gave me. And my deepest gratitude and love to Anna, who has given me all her unconditional support and love.
 

%\chapter*{Agradecimientos}
%
%Esta etapa que ha supuesto el doctorado ha sido una de las más gratificantes, completas y duras de mi vida. Como en una buena novela ha tenido un poco de todo: grandes retos, acción, aciertos, fracasos y sobre todo mucha reflexión. Reflexiones sobre la ciencia, el mundo científico, nuestro papel en el desarrollo de la sociedad a través de las ideas, la vida, el universo y todo lo demás. Soy consciente de la evolución que esta etapa ha producido en mi persona, y no soy menos consciente de la influencia que muchas personas han tenido sobre mí; para llegar a donde estoy; para llegar a percibir el mundo como lo percibo hoy. Sirva esto como pequeño reconocimiento a todas esas personas que han contribuido a dicha labor. Espero no olvidar a nadie.
%
%Permitanme empezar por aquellos academicamente cercanos, mis supervisores. Sin la ayuda de Angel Sappa, Julio Guerrero y Antonio M. López esta tesis nunca habría visto la luz. A Angel doy gracias por su apoyo incondicional, las discusiones y por la libertad que me ha otorgado. Antonio, gracias por dejarme formar parte de algo más grande, por haberme tenido en cuenta para los proyectos más duros y por tu ayuda en los peores momentos. Julio, para ti necesitaría páginas enteras, ya que tu involucración y ayuda en esta tesis han sido descomunales. Gracias por haber recorrido este camino conmigo y por haberme enseñado tanto; algunos de los mejores momentos han sido delante de tu pizarra. Me gustaría agradecer también a Daniel Ponsa, por sus opiniones e ideas, que han contribuido a mejorar este proyecto.
%
%Es también prioritario el dar las gracias a aquellos que me acogieron en sus centros de investigación, empezando por Christoph Stiller del MRT en KIT. Jose Manuel Alvarez y Richard Hartley por la oportunidad que me brindaron en NICTA. Pablo F. Alcantarilla y Bjorn Stenger por compartir su tiempo en Toshiba Research Europe conmigo. Watanabe-san y Okada-san por darme la oportunidad de participar en el proyecto Visconti en Toshiba Research and Development. Durante estas estancias e internships he de agradecer también a Andreas Geiger, Henning Lategahn, Mathieu Salzmann, Simon Stent, Riccardo Gherardi y Ankur Handa, que me abrieron los ojos a nuevas ideas y compartieron su tiempo conmigo.
%
%Mi más profunda gratitud hacia mis compañeros del Centro de Visión por Computador con los que he compartido muchos momentos. Empezando con todos aquellos con los que he tenido el placer de colaborar estrechamente, como David Vázquez, cuya tenacidad ha revivido el grupo y nos ha inspirado a todos; Biel, Jordi, Sergi, Laura y Joanna, que me han permitido enseñarles un poco de lo que sé a cambio de lo mucho que ellos me han enseñado a mí; y gracias a todo el equipo de SYNTHIA, en especial a Laura, Fran, Elias y Marc, por su esfuerzo y por haberme ayudado a recorrer este camino. También agradecer a mis compañeros del IRI, Victor Vaquero y Francesc Moreno por sus esfuerzos durante nuestras colaboraciones.
%
%He de agradecer también al personal de administración del CVC por toda la ayuda y simpatía brindada a lo largo de estos años. A Mari Carmen, Ana, Claire, Montse, Eva, Silvia, Mireia, Alexandra y Meritxell, muchas gracias. 
%
%Debo agradecer a aquellos que lograron que mi paso por el CVC fuese memorable. Gracias por aportar hospitalidad y humanidad: Marco Pedersoli, Pep Gonfaus, Javier Marín, Jordi Gonzalez, Xu Hu, Adela Barbulescu, Camp Davesa e Ivet Rafegas. Sin Felipe, Hana, Yaxing, Dena y Bojana el descanso de la comida no sería lo mismo. Y sobre todo gracias a mi círculo más cercano, Onur, Gemma, Ariel, Francesco, y Arash, que han estado ahí compartiendo conmigo alegrías y tristezas. Bien sabéis que nunca podré agradecer suficiente todo lo que Onur, Gemma y Ariel han hecho por mí en los momentos más oscuros. Vuestra amistad es incomparable.
%
%Dejo para el final a mi círculo más íntimo. Los agradecimientos a toda mi familia, a mis padres José Ángel y María Dolores, a mi hermana Marina y a mis abuelos, Carmen, Manuel, Pedro y Lola; gracias a los que estáis y a los que ya habéis marchado por haberme transmitido la fuerza y el valor para recorrer este camino. Y mi total gratitud a Anna, quien me ha brindado alegría junto a todo su apoyo incondicional y cariño. 

\setlength{\parskip}{1em}
