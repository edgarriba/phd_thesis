\section{Motivation}

\begin{figure}[!b]
    \centering
    \vspace{4mm}
    \includegraphics[width=1\textwidth]{header2.eps}
    %\vspace{-5mm}
    \caption[Example of Reinhard vs our approach in terms of artefacts.]{Example results for semantic labelling on CamVid computed on the original images \textbf{(top)} and after the proposed transformation \textbf{(bottom).}}
    \label{41:fig:header1}
    %\vspace{+0mm}
\end{figure}

In the context of driving scenes images are
taken from a mobile platform in uncontrolled, cluttered
environments, and the appearance of the objects greatly varies
depending on elements such as daytime, weather, illumination and
acquisition conditions~\cite{DaimlerECCV:2014}. Algorithms must be robust to the highly
dynamic nature of the environment and, due to the time constraints
of the applications, they require to be computationally efficient
to achieve real-time capabilities. In this chapter, we focus on making semantic segmentation algorithms more tolerant to the aforementioned variations of
the scene in an efficient and unsupervised fashion.

During the last decade, a number of approaches have focused on
providing accurate semantic labelling~\cite{Ladicky:2013,HoiemIJCV:2007,Make3dCVPR:2014,ZhangECCV:2010}.
These approaches exploit the continuously increasing available
data to learn more comprehensive models, able to deal with as many
situations as possible. Unfortunately, dealing with general urban
scenes involves a large number of different situations, making
difficult to foresee a dataset large-enough (\ie, complete) to
cover all possible conditions. Furthermore, it is clearly
infeasible to collect and learn in a single model all possible
situations. As a consequence, in practise, classifiers are trained
on relatively small datasets covering a limited number of
conditions and provide promising results when they are evaluated
on data from a similar distribution. However, as shown
in~Fig.~\ref{41:fig:header1} (\textbf{top}), there is a significant decrease in
performance when these algorithms are applied to images coming
from a different distribution. A common approach to solve this
problem consists of retraining the classifier with labelled
instances from each new environment (domain). More recent works
deal with this problem adapting the classifier to each new domain
transferring the knowledge by domain adaptation
techniques~\cite{XRV2013, Vazquez:2013b,LixinPAMI2012}. However,
these approaches require prior knowledge of the target domain and
the collection of labelled instances which is time consuming and
impractical for autonomous driving applications, where the domain
changes dynamically over time.


In contrast to these methods, here, we propose an efficient
unsupervised image transformation method following a global color
transfer strategy (Fig.~\ref{41:fig:header1} (\textbf{bottom})). In this way, test images are dynamically
transformed according to a set of reference images by imposing
their characteristic colors. This operation is applied following a
novel one-to-many scheme, which generalizes classical methods such
as Reinhard~\cite{Reinhard:2001}. The proposed scheme overcomes
the limitation of classical techniques when applied to dynamic
environments, since the quality of the results depends on how
similar are the images. Unlike classical methods, our algorithm
only assumes that the context of the image is known (urban scene)
and, thanks to the one-to-many transformation we can model the
diversity of the reference images (\ie, one showing sunny
conditions, the other showing bad weather, blooming, sky
saturations, etc.), to increase the chances of having a
reference and a test image with similar distributions. In
addition, our approach can naturally deal with the temporal
consistency of video streams to perform a coherent transformation.
All these elements serve to improve the performance on the side of
the  semantic labelling method, avoiding a cumbersome retraining
stage. Our algorithm does not require prior knowledge of the
domain nor labelled information and is, therefore, very suitable
for vehicle driving scenarios.


To demonstrate the benefits of our approach we conduct experiments
on three different semantic labelling frameworks (\ie,
Hoiem~\cite{HoiemIJCV:2007}, recent work by
Yang~\cite{Make3dCVPR:2014} and Darwin~\cite{DARWIN}), on two
publicly available datasets KITTI~\cite{KITTI} and
CamVid~\cite{CamVidBBDD:PRL2008}, along with a set of challenging
images collected from the Internet. Our extensive set of
experiments shows the effectiveness of our unsupervised
image transformation approach as a step of semantic
labelling at a negligible computational cost.
