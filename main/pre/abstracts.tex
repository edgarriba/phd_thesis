%\begingroup
%\let\cleardoublepage\clearpage



% English abstract
%\cleardoublepage
\chapter*{Abstract}

%\markboth{Abstract}{Abstract}
\addcontentsline{toc}{chapter}{Abstract (English/Spanish/Catalan)} % adds an entry to the table of contents
% put your text here
\vspace{-24mm}

From the early stages of Computer Vision, scene reconstruction has been one of the most studied topics leading to a wide variety of new discoveries and applications. Object grasping and manipulation, localization and mapping, or even visual effect generation are different examples of applications in which scene reconstruction has taken an important role for industries such as robotics, factory automation, or audio visual production. However, scene reconstruction is a huge topic that can be approached in many different ways which researchers already proposed quite effective solutions using classical Computer Vision methods. Formally, the problem of scene reconstruction can be formulated as a sequence, and sometimes independent processes that compose a pipeline. In this thesis, we analyse some of the parts of the reconstruction pipeline from which we will propose new methods using Convolutional Neural Networks (CNN) and approaching exotic solutions considering the optimisation of such methods in end to end fashion. Firstly, we review a classical approach for solving the reconstruction problem through local features by analyzing the detection and description of regions of interest. To be more precise, we review the state of the art of classical detectors and descriptors, and propose a couple of methods that inherently improve the existing results. It is a fact that Computer Science and software engineering are two fields that usually go hand in hand and evolve according to mutual needs making easier the design of complex and efficient algorithms. For this reason, we introduce a framework specifically designed to work with generic and classical Computer Vision techniques along with deep neural networks. In essence, we created a library that eases the implementation of complex pipelines for computer vision algorithms so that can be included within neural networks and be used to backpropagate gradients throw a common optimisation framework. Finally, in the last chapter of this thesis we develop the aforementioned concept of designing end to end systems with classical projective geometry. Thus, we propose a solution to the problem of synthetic view generation by hallucinating novel views from high deformable cloths objects using a geometry aware end to end system. To summarize, in this thesis we demonstrate that with a proper design and mixing classical geometric computer vision methods with deep learning techniques, the resulting end to end systems can improve by far the performance over existing methods.

\vspace{1mm}
\textbf{Key words:} \textit{Computer Vision, Scene Reconstruction, Local Features, Differentiable Operators, Views Synthesis Generation}

% German abstract

\begin{otherlanguage}{spanish}
%\cleardoublepage
\chapter*{Resumen}

\vspace{-24mm}

Escribir abstract en spanish (one page max)

\vspace{1mm}
\textbf{Palabras clave:} \textit{TBD}



%put your text here
\end{otherlanguage}



% French abstract
\begin{otherlanguage}{catalan}
\cleardoublepage
\chapter*{Resum}
\vspace{-24mm}

Escriure el abstract en catala (TBD)

\vspace{1mm}
\textbf{Paraules clau:} \textit{TBD}


\end{otherlanguage}

%\endgroup			
%\vfill
